%%%%%%%%%%%%%%%%%%%%%%%%%%%%%%%%%%%%%%%%%
% Programming/Coding Assignment
% LaTeX Template
%
% This template has been downloaded from:
% http://www.latextemplates.com
%
% Original author:
% Ted Pavlic (http://www.tedpavlic.com)
%
% Note:
% The \lipsum[#] commands throughout this template generate dummy text
% to fill the template out. These commands should all be removed when 
% writing assignment content.
%
% This template uses a Perl script as an example snippet of code, most other
% languages are also usable. Configure them in the "CODE INCLUSION 
% CONFIGURATION" section.
%
%%%%%%%%%%%%%%%%%%%%%%%%%%%%%%%%%%%%%%%%%

%----------------------------------------------------------------------------------------
%	PACKAGES AND OTHER DOCUMENT CONFIGURATIONS
%----------------------------------------------------------------------------------------

\documentclass{article}
\usepackage[utf8]{inputenc}
\usepackage[brazil]{babel}

\usepackage{amsmath,amsthm,amssymb,amsfonts} % Math stuff
%\usepackage{enumitem,tikz}
\usepackage{enumitem} % Better enumerates

\usepackage{fancyhdr} % Required for custom headers
\usepackage{lastpage} % Required to determine the last page for the footer
\usepackage{extramarks} % Required for headers and footers
\usepackage[usenames,dvipsnames]{color} % Required for custom colors
\usepackage{graphicx} % Required to insert images
\usepackage{listings} % Required for insertion of code
\usepackage{courier} % Required for the courier font
\usepackage{lipsum} % Used for inserting dummy 'Lorem ipsum' text into the template

% Margins
\topmargin=-0.45in
\evensidemargin=0in
\oddsidemargin=0in
\textwidth=6.5in
\textheight=9.0in
\headsep=0.25in

\linespread{1.1} % Line spacing

% Set up the header and footer
\pagestyle{fancy}
\chead{\hmwkClass: \hmwkTitle} % Top center head
\rhead{\firstxmark} % Top right header
\lfoot{\lastxmark} % Bottom left footer
\cfoot{} % Bottom center footer
\rfoot{Page\ \thepage\ of\ \protect\pageref{LastPage}} % Bottom right footer
\renewcommand\headrulewidth{0.4pt} % Size of the header rule
\renewcommand\footrulewidth{0.4pt} % Size of the footer rule

\setlength\parindent{0pt} % Removes all indentation from paragraphs

%----------------------------------------------------------------------------------------
%	DOCUMENT STRUCTURE COMMANDS
%	Skip this unless you know what you're doing
%----------------------------------------------------------------------------------------

% Header and footer for when a page split occurs within a problem environment
\newcommand{\enterProblemHeader}[1]{
\nobreak\extramarks{#1}{#1 continued on next page\ldots}\nobreak
\nobreak\extramarks{#1 (continued)}{#1 continued on next page\ldots}\nobreak
}

% Header and footer for when a page split occurs between problem environments
\newcommand{\exitProblemHeader}[1]{
\nobreak\extramarks{#1 (continued)}{#1 continued on next page\ldots}\nobreak
\nobreak\extramarks{#1}{}\nobreak
}

\setcounter{secnumdepth}{0} % Removes default section numbers
\newcounter{homeworkProblemCounter} % Creates a counter to keep track of the number of problems

\newcommand{\homeworkProblemName}{}
\newenvironment{homeworkProblem}[1][\unskip]{ % Sets homework environment with adittional argument for extra naming
    \stepcounter{homeworkProblemCounter} % Increase counter for number of problems
    \renewcommand{\homeworkProblemName}{Questão \arabic{homeworkProblemCounter} #1} % Assign \homeworkProblemName the name of the problem
    \subsection{\homeworkProblemName} % Make a section in the document with the custom problem count
    \enterProblemHeader{\homeworkProblemName} % Header and footer within the environment
}{
\exitProblemHeader{\homeworkProblemName} % Header and footer after the environment
}

\newcommand{\problemAnswer}[1]{ % Defines the problem answer command with the content as the only argument
\noindent\framebox[\columnwidth][c]{\begin{minipage}{0.98\columnwidth}#1\end{minipage}} % Makes the box around the problem answer and puts the content inside
}

\newcommand{\homeworkSectionName}{}
\newenvironment{homeworkSection}[1]{ % New environment for sections within homework problems, takes 1 argument - the name of the section
\renewcommand{\homeworkSectionName}{#1} % Assign \homeworkSectionName to the name of the section from the environment argument
\subsection{\homeworkSectionName} % Make a subsection with the custom name of the subsection
\enterProblemHeader{\homeworkProblemName\ [\homeworkSectionName]} % Header and footer within the environment
}{
\enterProblemHeader{\homeworkProblemName} % Header and footer after the environment
}

%----------------------------------------------------------------------------------------
%	NAME AND CLASS SECTION
%----------------------------------------------------------------------------------------

\newcommand{\hmwkTitle}{Relatório EP 1} % Assignment title
\newcommand{\hmwkDueDate}{18 de Setembro de 2016} % Due date
\newcommand{\hmwkClass}{MAC0210} % Course/class

%----------------------------------------------------------------------------------------
%	TITLE PAGE
%----------------------------------------------------------------------------------------

\title{
\vspace{2in}
\textmd{\textbf{\hmwkClass:\ \hmwkTitle}}\\
\normalsize\vspace{0.1in}\small{\hmwkDueDate}\\
\vspace{3in}
}

\author{\textbf{Nathan Benedetto Proença - 8941276}\\
\textbf{Victor Sena Molero - 8941317}}
\date{} % Insert date here if you want it to appear below your name

%----------------------------------------------------------------------------------------

\begin{document}

\maketitle

%----------------------------------------------------------------------------------------
%	TABLE OF CONTENTS
%----------------------------------------------------------------------------------------

%\setcounter{tocdepth}{1} % Uncomment this line if you don't want subsections listed in the ToC

\newpage
\tableofcontents
\newpage

\section{Parte 1: Aritmética de Ponto Flutuante}

\begin{homeworkProblem}[(3.11)]
Suponha que temos um sistema de representação de ponto flutuante com base 2 e,  
$$x = \pm S \times 2^{E} \text{, }$$
$$\text{com }  S = (0.1b_2b_3b_4\dots b_{24}) \text{, } $$
$$\text{i.e, }  \frac{1}{2} \leq S < 1$$
onde o expoente $-128 < E < 127$.

\begin{enumerate}[label={\alph*)}]
    \item Qual é o maior número de ponto flutuante desse sistema?
        \begin{proof}[Resposta]
        $2^{126} - 2^{101}$, basta preencher todos os bits (de $b_2$ até $b_{24}$) e escolher o maior expoente possível.
        \end{proof}
    \item Qual é o menor número de ponto flutuante positivo desse sistema?
        \begin{proof}[Resposta]
        $2^{-128}$, basta escolher a menor mantissa possível ($0.1$) e o menor expoente possível ($-127$).
        \end{proof}
    \item Qual é o menor inteiro positivo que não é exatamente representável nesse sistema?
        \begin{proof}[Resposta]
        $2^{24} + 1$, basta escolher a menor mantissa não representável ($0.10\dots 01$) e o menor expoente para o qual ela representa um inteiro ($25$).
        \end{proof}
\end{enumerate}
\end{homeworkProblem}

\begin{homeworkProblem}[(5.1)]
Qual é a representação do número $1/10$ no formato IEEE single para cada um dos quatro modos de arredondamento?
\begin{proof}[Resposta]
A resposta curta é: 
$$x = 0.0\overline{0011} \text{, }$$
$$\mathrm{round\_down}(x) = \mathrm{round\_towards\_zero}(x) = 1.10011001100110011001100 \times 2^{-4} \text{ e }$$ 
$$\mathrm{round\_up}(x) = \mathrm{round\_to\_nearest}(x) = 1.10011001100110011001101 \times 2^{-4} \text{.}$$

Se $x$ é uma representação binária exata de $1/10$, $x = 0.0\overline{0011} = 1.\overline{1001} \times 2^{-4}$ (os números abaixo da barra representam uma dízima periódica, se repetem infinitamente).

Calculamos então $x_- = 1.10011001100110011001100 \times 2^{-4}$ e $x_+ = 1.10011001100110011001101 \times 2^{-4}$. Se o modo é \textit{Round Down}, o número será representado por $x_-$ e se for \textit{Round Up}, será $x_+$, ambos pela definição dos modos.

Já que $x > 0$, o modo $\textit{Round towards zero}$ também usará $x_-$, porém $x$ é mais próximo de $x_+$ do que de $x_-$, basta perceber que o erro relativo entre $x$ e $x_-$ é maior que $1/2$, portanto, o modo $\textit{Round to nearest}$ usará a representação $x_+$.
\end{proof}
E para os números $1 + 2^{-25}$
\begin{proof}[Resposta]
$$x = 1.0000000000000000000000001 \times 2^0\text{, }$$
$$\mathrm{round\_down}(x) = \mathrm{round\_towards\_zero}(x) = \mathrm{round\_to\_nearest}(x) = 1.00000000000000000000000 \times 2^0\text{, }$$
$$\mathrm{round\_up}(x) = 1.00000000000000000000001 \times 2^0\text{.}$$
Se $x$ é uma representação binária exata de $1+2^{-25}$, $x = 1.0000000000000000000000001 \times 2^0$, $x_- = 1.00000000000000000000000 \times 2^0$ e $x_+ = 1.00000000000000000000001 \times 2^0$. O modo \textit{Round down} vai levar para $x_-$ e o modo \textit{Round up} vai levar para $x_+$, como sempre.  

Já que $x > 0$, o modo $\textit{Round towards zero}$ também usará $x_-$, além disso, o erro relativo entre $x$ e $x_-$ é $1/4$, logo, o modo $\textit{Round to nearest}$ também levará para $x_-$.
\end{proof}
e $2^{130}$?
\begin{proof}[Resposta]
$$x = 1 \times 2^{130} \text{, }$$
\begin{equation*} \begin{split}
    \mathrm{round\_down}(x) = \mathrm{round\_towards\_zero}(x) = \mathrm{round\_to\_nearest}(x) = \\
    = \mathrm{round\_up}(x) = 1.00000000000000000000000 \times 2^{130}\text{.}
\end{split} \end{equation*}
Se $x$ é uma representação binária exata de $2^{130}$, $x = 1 \times 2^{130}$, porém, $x$ é exatamente representável no sistema IEEE single, logo, em todos os modos de arredondamento, será representado como $1.00000000000000000000000 \times 2^{130}$.
\end{proof}
\end{homeworkProblem}

\begin{homeworkProblem}[(6.4)]
Qual é o maior número de ponto flutuante $x$ tal que $1 \oplus x$ é exatamente 1, assumindo que o formato usado é IEEE single e modo de arredondamento para o mais próximo?

\begin{proof}[Resposta]
$x = 2^{-24}$. $1 + x$ é igualmente próximo de $1 + 2^{-23}$ e $1$, porém, por causa do critério de arredondamento em empate (0 menos significativo), ele é arredondado para $1$, qualquer $x$ maior do que esse causará um arredondamento para um número maior do que $1$.
\end{proof}

E se o formato for IEEE double?

\begin{proof}[Resposta]
$x = 2^{-53}$, seguindo a mesma lógica usada para concluir a resposta do item anterior.
\end{proof}
\end{homeworkProblem}

\begin{homeworkProblem}[(6.8)]
Em aritmética exata, a soma é um operador comutativo e associativo. O operador de soma de ponto flutuante é comutativo?

\begin{proof}[Resposta]
Sim, pois para calcular o resultado em soma de ponto flutuante o padrão exige que seja calculado o valor exato e, então, arredondado para o sistema escolhido. Formalmente, denotaremos por $fl(x)$ a representação em ponto flutuante de um real $x$ e por $\oplus$ a operação de soma em ponto flutuante. Temos $fl(x) \oplus fl(y) = fl(fl(x) + fl(y)) = fl(fl(y) + fl(x)) = fl(y) \oplus fl(x)$.
\end{proof}

E associativo?

\begin{proof}[Resposta]
Não, os erros de arredondamento podem fazer com que a ordem das somas faça diferença. Por exemplo, considere um sistema com um dígito binário de precisão ($1.b_1$) e expoentes entre $-4$ e $4$, por exemplo. Escolha os números $x = 1 = 2^0$ e $y = z = 1/4 = 2^{-2}$. Teremos, na notação do sistema (base binária) que $(x \oplus y) \oplus z = (1.0 \times 2^0 \oplus 1.0 \times 2^{-2}) \oplus 1.0 \times 2^{-2} = 1.0 \times 2^0 \oplus 1.0 \times 2^{-2} = 1.0 \times 2^0$, por outro lado, $x \oplus (y \oplus z) = 1.0 \times 2^0 \oplus 1.0 \times 2^{-1} = 1.1 \times 2^0$.
\end{proof}
\end{homeworkProblem}

\end{document}
